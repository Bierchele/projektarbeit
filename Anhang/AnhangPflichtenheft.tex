\subsection{Pflichtenheft (Auszug)}
\label{app:Pflichtenheft}

\subsubsection*{Zielbestimmung}
Das vorliegende Pflichtenheft stellt die spezifischen Anforderungen an die Anwendung ``SupportBot'' dar. Diese Anforderungen dienen als Grundlage für die Implementierung und Entwicklung der Anwendung.
		

\begin{enumerate}[itemsep=0em,partopsep=0em,parsep=0em,topsep=0em]
\item Musskriterien
	\begin{enumerate}
		\item Ticket-Einstufung und Kompetenzbewertung
		\begin{itemize}
			\item Die Anwendung muss in der Lage sein, Support-Tickets eigenständig zu analysieren und korrekt in bestimmte Kategorien oder Schwierigkeitsgrade einzustufen. Dies dient dazu, die Priorität und Bearbeitungshierarchie festzulegen.
			\item Die Anwendung sollte die eigene Kompetenz bei der Lösung von Support-Tickets bewerten und sicherstellen, dass sie nur Tickets bearbeitet, für die sie ausreichend qualifiziert ist. Dies gewährleistet eine effiziente Ticketzuweisung.
		\end{itemize}
		\item Zugriff auf E-Mail-Vorlagen
			\begin{itemize}
			\item Die Anwendung sollte in der Lage sein, auf eine Vielzahl von vordefinierten E-Mail-Vorlagen für den Kunden-Support zuzugreifen. Bei Bedarf kann sie diese Vorlagen in die Kommunikation mit den Kunden einbinden. Dies ermöglicht eine konsistente und effektive Kommunikation.
			\end{itemize}
		\item Verhaltenstransparenz und Export
			\begin{itemize}
			\item Die Anwendung sollte die Fähigkeit besitzen, ihre Aktionen und Entscheidungen in einem exportierbaren Format bereitzustellen. Dies dient der Nachvollziehbarkeit und Analyse des Bot-Verhaltens. Das exportierbare Format sollte für Analysezwecke geeignet sein.
			\end{itemize}
		\item Bedienbarkeit
			\begin{itemize}
			\item Die Anwendung wird als ausführbares Programm bereitgestellt. Dies ermöglicht die einfache Installation und Nutzung auf den Rechnern der Kundensupport-Mitarbeiter.
			\end{itemize}
	\end{enumerate}
\end{enumerate}


