% !TEX root = ../Projektdokumentation.tex
\section{Implementierungsphase} 
\label{sec:Implementierungsphase}

\subsection{Implementierung des Generativen Transformers}
\label{sec:ImplementierungGenerativenTransformers}
Die Implementierung des Generativen Transformers erfolgte unter Verwendung des Klassendiagramms 
aus Kapitel~\ref{sec:Architekturdesign}: \nameref{sec:Architekturdesign} als Vorlage.
Der erste Schritt bestand darin, die lokale Konfiguration des \ac{LLaMA}-2 Modells durchzuführen. 
Um diese Konfiguration zu starten, musste das Modell zunächst heruntergeladen werden.
Hierfür wurde das Repository von Meta geklont und nach der Registrierung auf der Webseite wurde ein 
Freischaltungslink erhalten. Dieser Link wurde als Parameter für ein Bash-Skript verwendet, 
um das gewünschte Modell auszuwählen und herunterzuladen.

\paragraph{Die folgenden Schritte umfassten:} 
\begin{itemize}
	\item Die Konvertierung des Modells in das \ac{GGUF}-Format.
	\item Die Quantifizierung des Modells.
\end{itemize}

Mithilfe der \ac{LLaMA}.cpp-Bibliothek war es möglich, das Modell mithilfe der convert.py-Funktion
in das Zielformat GGUF zu konvertieren. Dieses Format ist ein generalisiertes Format für \ac{LLM}s 
und bietet den Vorteil, dass ein \ac{LLM} nur mit einer CPU betrieben werden kann. 
Aufgrund begrenzter Hardware-Ressourcen mussten die Tokens des \ac{LLaMA}-2-Modells gekürzt werden. 
Tokens sind Wortkombinationen, die mit einer bestimmten Wahrscheinlichkeit aneinandergereiht werden. 
Standardmäßig haben Tokens eine Länge von 16 Bit. Mit der quantize-Funktion von \ac{LLaMA}.cpp wurden 
diese auf 4 Bits gekürzt, um eine schnellere Verarbeitung der Eingabe zu ermöglichen, 
jedoch auf Kosten der Genauigkeit der Ausgabe.





\subsection{Anbindung an die Zendesk Schnittstelle}
\label{sec:ImplementierungBenutzeroberflaeche}
In diesem Schritt wurde eine Verbindung zur Zendesk-Anwendungsschnittstelle hergestellt.
Es gab verschiedene Möglichkeiten zur Authentifizierung. 
Anstelle von Benutzername und Passwort wurde sich für die Verwendung eines \ac{API}-Tokens entschieden.
Die Wahl des \ac{API}-Tokens erschien in diesem Fall sicherer, da dieser Token ausschließlich
für diese Anwendung gilt, somit wird der Branchenstandard für Machine to Machine Kommunikation gewährleistet.
Durch die Kombination von Benutzername und \ac{API}-Token ist es möglich, im Namen des Nutzers mit der 
Anwendungsschnittstelle zu interagieren. Dies ermöglicht die 
Überwachung der Support-Bots, da bei jeder Anfrage an die \ac{API} die
erforderlichen Informationen mitgesendet werden.

Ein Screenshot der implentierten Klasse befindet sich im \Anhang{Screenshots}.

\subsection{Implementierung der Geschäftslogik}
\label{sec:ImplementierungGeschaeftslogik}
Die Geschäftslogik konzentriert sich hauptsächlich auf die Steuerung des Datenflusses
zwischen der Zendesk-Anwendungsschnittstelle und dem \ac{LLaMA}-2-Modell.
Ein Anwendungszyklus beginnt mit der Abfrage der Zendesk-Schnittstelle, 
die im ZendeskService durchgeführt wird, um die Daten eines Support-Tickets abzurufen. ~\\
Diese Daten werden anschließend in eine SupportTicket-Klasse umgewandelt. ~\\
Der Inhalt der Support-Ticket-Klassen wird mithilfe von LangChain 
in eine Vorlagen-Eingabe eingebettet. Diese Eingabe fordert das \ac{LLaMA}-Modell auf, 
als Support-Mitarbeiter zu agieren und auf bestimmte Schlüsselwörter zu achten.
Zum Beispiel, wenn das Ticket die Beschreibung ``E-Mail verloren, Neue Tickets'' enthält.
Wird nun das Schlüsselwort \texttt{RESEND\_TICKET} erwartet. ~\\ Diese Logik wird in der 
\texttt{classify\_tickets-Methode} der SupportAgent-Klasse umgesetzt. ~\\
Nach der Klassifizierung des Inhalts des Support-Tickets wird das
classification-Attribut ~\\ der SupportTicket-Klasse auf das entsprechende Schlüsselwort 
gesetzt. In einem weiteren Schritt wird dieses Attribut ausgelesen, 
um das passende Makro von der Zendesk-Anwendungsschnittstelle aufzurufen. 
Die Antwort des Aufrufs wird dann in die Payload der \texttt{reply\_to\_customer-Methode}
der ZendeskService-Klasse eingebettet, um eine Antwort für das Support-Ticket
zu generieren. Dabei wird das \texttt{ticket\_id-Attribut} verwendet, 
um die richtige Antwort dem entsprechenden Ticket zuzuordnen.

\paragraph{Auschnitt}
Ein Auschnitt der SupportAgent-Klasse befindet sich im \Anhang{Screenshots}.
