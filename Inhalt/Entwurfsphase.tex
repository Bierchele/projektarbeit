% !TEX root = ../Projektdokumentation.tex
\section{Entwurfsphase} 
\label{sec:Entwurfsphase}

\subsection{Zielplattform}
\label{sec:Zielplattform}
Bei Wahl der Zielplattform gab es einige Auswahlkriterien zu beachten, um die Implementierung zu erleichtern.
Diese teilen sich in folgende Punkte auf.

\paragraph{Schlankes und leistungsstarkes \ac{GPT}-Modell:} 
\begin{itemize}
	\item Die gewählte Zielplattform erforderte ein \ac{GPT} -Modell, das in der Lage ist, mit begrenzten Ressourcen in angemessener Zeit qualitativ hochwertige Antworten zu generieren.
\end{itemize}

\paragraph{Freie kommerzielle Nutzung des \ac{GPT}-Modells:} 
\begin{itemize}
	\item Die Möglichkeit, das \ac{GPT}-Modell kommerziell frei zu nutzen, ist ein wichtiger Faktor, um Lizenzbeschränkungen zu vermeiden.
\end{itemize}

\paragraph{Kompatibilität mit einem lokalen \ac{GPT}-Modell:} 
\begin{itemize}
	\item Die reibungslose Integration mit einem lokalen \ac{GPT}-Modell ist von entscheidender Bedeutung, um die Anforderungen mit dem fachgerechten Umgang mit personenbezogenen Daten, des Projekts zu erfüllen.
\end{itemize}

\paragraph{Kompatibilität mit der genutzten LangChain-Bibliothek:} 
\begin{itemize}
	\item Da die Kernlogik des Projekts hauptsächlich auf dem Datenaustausch zwischen dem lokalen \ac{GPT} -Modell und der Zendesk-\ac{API} basiert, war die Kompatibilität mit der LangChain-Bibliothek von großer Relevanz, da diese Bibliothek die Datem anreichert.
\end{itemize}

\paragraph{Erstellung eines ausführbaren Programms für macOS:} 
\begin{itemize}
	\item Die Zielplattform sollte die Möglichkeit bieten, ein ausführbares Programm für macOS zu generieren, um sicherzustellen, dass die Anwendung in der gewünschten Umgebung ausgeführt werden kann.
\end{itemize}

Basierend auf diesen Kriterien wurde die Entscheidung für das 7B \ac{LLaMA}-2 Modell
von Meta als Zielplattform getroffen. Dieses Modell zeichnet sich durch seine 
kompakte Größe von 13 \ac{GB} aus und kann auf einem Rechner mit lediglich einer \ac{CPU}
und 16 \ac{GB} \ac{RAM} effizient betrieben werden.
Die Auswahl der Programmiersprache wurde stark von der Kompatibilität mit dem \ac{GPT}-Modell 
beeinflusst. Python wurde als die geeignete Programmiersprache gewählt, da sie eine 
einfache Anbindung an die LangChain-Bibliothek ermöglicht, die wiederum die Verwaltung
des \ac{GPT}-Modells erleichtert. Darüber hinaus bietet Python einen einfachen Weg, um 
plattformunabhängige ausführbare Programme zu generieren.


\subsection{Architekturdesign}
\label{sec:Architekturdesign}
Es wird entschieden eine objektorientierte
Architektur zu verwenden, um die Anwendung in sinnvolle Entitäten zu strukturieren.
Dies ermöglicht eine bessere Wartbarkeit und Erweiterbarkeit der einzelnen Funktionen,
da sie mit geringeren Abhängigkeiten voneinander entwickelt werden können.
Eine weitere wesentliche Entscheidung bestand darin, die LangChain-Bibliothek zu verwenden.
Diese Bibliothek erfüllt mehrere wichtige Funktionen in der Architektur. 
Zum einen ermöglicht sie die Anreicherung der Eingabe für das \ac{LLaMA}-2 Modell.
Darüber hinaus bietet sie die Möglichkeit zur Verwaltung verschiedener Modelle. 
Die Einbindung der LangChain Bibliothek spielt eine entscheidende Rolle, da sie dazu beiträgt, 
den Inhalt der Support-Tickets effizienter zu verarbeiten.
Eine ausführliche Auflistung der Klassen und deren Zusammenwirken finden Sie im \Anhang{app:Klassendiagramm}


\subsection{Entwurf der Benutzeroberfläche}
\label{sec:Benutzeroberflaeche} 
Für diese Anwendung wurde auf eine Benutzeroberfläche verzichtet.


\subsection{Datenmodell}
\label{sec:Datenmodell}
Die Anwendung verwendet keine direkte Datenbankanbindung. 
Da der Kundensupport bereits die Zendesk-Software für alle relevanten
Anfragen verwendet, wird diese für das Überwachen und Persistieren  der 
erforderlichen Daten genutzt. Die Anwendung greift auf die Daten in 
Zendesk zu, ohne separate Datenbanken zu verwenden.





\subsection{Geschäftslogik}
\label{sec:Geschaeftslogik}
Die Geschäftslogik der Anwendung besteht aus zwei zustandslosen 
Entitäten, die jeweils als Wrapper für die Zendesk-Schnittstelle und dem \ac{LLaMA}-2-Modell fungieren. Diese Logik 
teilt sich in den Datenaustausch mit der Zendesk-\ac{API} und die 
Anreicherung von Texten mit dem \ac{LLaMA}-2-Modell auf.
Der Prozess beginnt damit, Support-Tickets von der Zendesk-\ac{API} 
abzurufen und die ausgewählten Eigenschaften in Support-Ticket-Entitäten 
umzuwandeln. Als nächster Schritt erfolgt die Klassifizierung der Tickets,
um das passende Makro aus Zendesk anzuwenden. Dieses Makro
wird in einem POST-Request an die Zendesk-\ac{API} gesendet.




\subsection{Maßnahmen zur Qualitätssicherung}
Um die Qualität der Anwendung zu gewährleisten und einen reibungslosen Betrieb
sicherzustellen, wurden Unit-Tests und Integrationstests implentiert. 
Diese Tests ermöglichen die Überprüfung und Sicherstellung 
der Kernlogik sowohl in ihren einzelnen Komponenten als auch als 
Gesamtsystem ordnungsgemäß funktioniert.
Dies trägt zur besseren Wartbarkeit und Erweiterbarkeit der Anwendung bei.


\subsection{Pflichtenheft}
Der \Anhang{app:Pflichtenheft} umfasst alle technischen Kriterien, die in der Anwendung
umgesetzt werden müssen. Es entstand aufgrund der Arbeitsergebnisse der 
Entwurfsphase. Dieses Dokument ist von entscheidender Bedeutung für
die präzise Planung der Implementierungsphase, da es Aufgaben definiert, die zeitlich gut einschätzbar sind.