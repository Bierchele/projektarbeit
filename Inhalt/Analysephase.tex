% !TEX root = ../Projektdokumentation.tex
\section{Analysephase} 
\label{sec:Analysephase}


\subsection{Ist-Analyse} 
\label{sec:IstAnalyse}
Der Ist-Zustand, wie bereits im Abschnitt 1.1 ``Projektumfeld'' beschrieben, 
zeigt, dass im Kunden-Support vermehrt triviale und repetitive Kundenanfragen eingehen. 
Diese Anfragen bestehen oft aus dem erneuten Verschicken von E-Mails mit gekauften Tickets, 
dem Löschen von personenbezogenen Daten und der Umpersonalisierung von 
ticketbezogenen Informationen. Der Kunden-Support bedient sich dabei der Webanwendung Zendesk, 
welche die Nutzung von Makros ermöglicht. Makros sind vorgefertigte E-Mail-Vorlagen, 
die bereits den passenden Text und gegebenenfalls Links enthalten.

Die aktuelle Lage ergibt folgende Probleme:

\begin{itemize}
	\item Manuelle Verknüpfung von Support-Ticket und Makro: Ein Support-Mitarbeiter muss den Inhalt eines Support-Tickets stets manuell mit dem passenden Makro verknüpfen, was zeitaufwändig und fehleranfällig ist.
	\item Mangelnde Klassifizierung und automatische Anwendung: Zendesk bietet keine Möglichkeit, den Inhalt zu klassifizieren und das Makro eigenständig anzuwenden, was die Effizienz einschränkt.
	\item Beeinträchtigung der Effizienz: Aufgrund der Menge dieser Anfragen werden andere Fälle vernachlässigt, hinzu kommt die zeitliche Begrenzung eines Mitarbeiters. Dies beeinflusst die Gesamteffizienz des Kunden-Supports und kann zu unzufriedenen Kunden führen.
\end{itemize}




\subsection{Wirtschaftlichkeitsanalyse}
\label{sec:Wirtschaftlichkeitsanalyse}

Die Teilautomatisierung der in Abschnitt 3.1 ``Ist-Analyse'' beschriebenen Problemstellung bietet
die Möglichkeit, die Wirtschaftlichkeit dieses Vorhabens zu validieren. Um dies zu erreichen,
wird eine Prozesskostenrechnung durchgeführt, um die ungefähren Kosten der Bearbeitung eines 
Support-Tickets zu ermitteln.

Das Hauptziel dieser Wirtschaftlichkeitsanalyse besteht darin, die Amortisationsdauer des 
Projekts zu bestimmen. Dies bedeutet, dass wir die erwarteten Projektkosten aus Abschnitt 3.3 
``Projektkosten'' mit den eingesparten Kosten durch die Teilautomatisierung gegenüberstellen 
werden.

Die Prozesskostenrechnung ermöglicht es, die finanziellen Auswirkungen der 
Implementierung der Lösung zu bewerten und festzustellen, 
ob das Projekt in einem angemessenen Zeitrahmen Rendite erzielt.




\subsubsection{Projektkosten}
\label{sec:Projektkosten}

\paragraph{Projekkostenrechnung} Die Kosten für das Projekt setzen sich aus verschiedenen Komponenten zusammen:


\paragraph{Personalkosten:} 
\begin{itemize}
	\item Als Auszubildender im dritten Lehrjahr beträgt mein Bruttogehalt 1145€ pro Monat.
	\item Für die anderen Mitarbeiter wird pauschal ein Stundenlohn von 20€ angenommen.
\end{itemize}

\paragraph {Ressourcenkosten:}
\begin{itemize}
	\item 
	Die allgemeinen Ressourcenkosten, einschließlich der Bürofläche und der Hardware, 
   werden mit einem ungefähren Stundensatz von 15€ veranschlagt.
\end{itemize}


\begin{eqnarray}
8 \mbox{ h/Tag} \cdot 220 \mbox{ Tage/Jahr} = 1760 \mbox{ h/Jahr}\\
\eur{1250}\mbox{/Monat} \cdot 12 \mbox{ Monate/Jahr} = \eur{13745,6} \mbox{/Jahr}\\
\frac{\eur{13745,6} \mbox{/Jahr}}{1760 \mbox{ h/Jahr}} \approx \eur{7,81}\mbox{/h}
\end{eqnarray}

Mit den angegebenen Komponenten und einer Projektdurchführungszeit von 70 Stunden
belaufen sich die Projektkosten auf insgesamt \eur{2086,70}. In der unteren Tabelle~\ref{tab:Kostenaufstellung}
\tabelle{Kostenaufstellung}{tab:Kostenaufstellung}{Kostenaufstellung.tex}
befindet sich eine Veranschaulichung der einzelenen Kostenträger.

\subsubsection{Amortisationsdauer}
\label{sec:Amortisationsdauer} 

Um die Amortisationsdauer zu bestimmen wurde mithilfe einer Prozesskostenrechnung
der ungefähre Preis der Bearbeitung eines Tickets berechnet. 

Die Komponenten der Berechnung bestehen aus:

\paragraph {Eintrittshäufigkeit einer Aktivität:} 
\begin{itemize}
	\item Support-Ticket kategorisieren $  \approx  $ 10 \%  
	\item Lösung recherchieren $  \approx  $ 20 \%
	\item Kundenkorrespondenz $  \approx  $ 30 \%         	
\end{itemize}

\paragraph{Mitarbeiter Effizienz:} 
\begin{itemize}
	\item Jahresarbeitslohn  $  \approx  $ \eur{32.000,00} 
	\item Jahresarbeitsminuten  $  \approx  $ 110min   	
\end{itemize}

\paragraph{Prozessmenge und neutrale Kosten:} 
\begin{itemize}
	\item Prozessmenge  $  \approx  $  38.000 Support-Tickets pro Jahr   
	\item Leistungsmengenneutrale Kosten  $  \approx  $ 15\% 	
\end{itemize}

Verrechnet man diese Komponenten ergibt sich ein
ungefährer Preis von \eur{2,06} pro Support-Ticket.
Eine detaillierte Rechnung befindet sich im \Anhang{app:Prozesskosten}.

Entnimmt man dem \Anhang{app:Amortisationsrechnung} die ungefähren laufenden Betriebtskosten des Projekts von
\eur{0,40} und der geschätzen Arbeitsleistung von 10 Support-Tickets pro Tag 
kommt man auf eine Amortisationsdauer von 103 Tage
was ungefähr 3,3 Monaten entspricht. 
Die genaue Aufstellung kann dem genannten Anhang entnommen werden.


\subsubsection{\gqq{Make or Buy}-Entscheidung}
\label{sec:MakeOrBuyEntscheidung}
Die Entscheidung, keine alternative Software in Betracht zu ziehen und stattdessen 
die Entwicklung mit \ac{GPT}-Modellen fortzusetzen, basierte auf mehreren Faktoren. 
Die Sensibilität von personenbezogenen Daten und die Notwendigkeit, 
strenge Datenschutz- und Sicherheitsrichtlinien 
einzuhalten, erforderten eine Lösung, die diesen Anforderungen gerecht wird.
Die Integration der entwickelten Lösung mit Zendesk und die Sicherstellung der Einhaltung 
dieser Richtlinien waren von entscheidender Bedeutung. 
Es gab keine vorhandene Software, die sowohl die erforderlichen Datenschutzstandards 
erfüllte als auch nahtlos mit Zendesk integriert werden konnte
und gleichzeitig einen niedrigeren wirtschaftlichen Aufwand mit sich brachte.


\subsection{Anwendungsfälle}
\label{sec:Anwendungsfaelle}
Das im \Anhang{app:UseCase} bildet 
die zentralen Anwendungsfälle ab, die 
Interaktionen zwischen Benutzern, Fremdsystemen und der Anwendung veranschaulichen. 
Diese Anwendungsfälle werden während der Implementierung in die einzelnen 
Features und Funktionen der Anwendung aufgeschlüsselt.


\subsection{Qualitätsanforderungen}
\label{sec:Qualitaetsanforderungen}
Um die Qualität der Anwendung zu gewährleisten und Fehlerquellen zu minimieren,
wurden bestimmte Schlüsselwörter festgelegt, die von der Anwendung zur
Klassifizierung von Support-Tickets verwendet werden.
Ein besonders wichtiger Aspekt dieser Qualitätsanforderungen besteht darin, 
dass die Anwendung in der Lage sein muss, Halluzinationen oder 
fehlerhafte Klassifizierungen des \ac{GPT}-Modelles zu verhindern. 


\subsection{Lastenheft/Fachkonzept}
In zusammenarbeit mit den Stakeholder haben sich im \Anhang{app:Lastenheft} 
die sich darin befindlichen Anforderungen an die Anwendung herauskristallisiert.
\clearpage