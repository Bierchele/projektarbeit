% !TEX root = ../Projektdokumentation.tex
\section{Projektplanung} 
\label{sec:Projektplanung}

\subsection{Projektphasen}
\label{sec:Projektphasen}

Die Umsetzung dieses Projekts erfolgte innerhalb eines effektiven Zeitraums von 70 Stunden.
Um die Projektumsetzung optimal zu organisieren und den Zeitrahmen besser zu gliedern, 
habe ich mich für die Phaseneinteilung des erweiterten Wasserfallmodells entschieden. 
Diese Methode ermöglichte es mir, die Zeiteinteilung präziser zu gestalten, 
indem ich Teilaufgaben den verschiedenen Phasen des Projekts zugeordnet habe.
Die Phaseneinteilung half dabei, das Projekt in klare Abschnitte zu unterteilen 
und die Fortschritte zu verfolgen. 

\paragraph{Zeitplanung}
Auschnitt der groben Zeiteinschätzung.
\tabelle{Zeitplanung}{tab:Zeitplanung}{ZeitplanungKurz}\\
Eine ausführliche Zeitplanung findet sich im \Anhang{app:Zeitplanung}.


\subsection{Abweichungen vom Projektantrag}
\label{sec:AbweichungenProjektantrag}
\begin{itemize}
	\item Aufgrund von Ressourcen- und Zeitmangel musste auf ein Fine-Tuning des Modells verzichtet werden.
	\item Durch eine präzise Beschreibung der Klassen, Funktion, Variablen .etc wurde auf Kommentare im Quellcode weitgehen verzichtet.
	\item Da die Anwendung als eine Application Bundle bereit gestellt wird, wurde auf eine Benutzerdokumentation verzichtet, da die Anwendung als ausführbares Programm verfügbar ist.
\end{itemize}


\subsection{Ressourcenplanung}
\label{sec:Ressourcenplanung}

Die Ressourcenplanung umfasst alle direkten und indirekten Ressourcen. 
Dies beinhaltet das Personal, 
Hardware, Software und die von unserem Arbeitgeber 
bereitgestellte Bürofläche. Um die Kosten zu minimieren, 
wurde hauptsächlich auf Open-Source-Software zurückgegriffen, 
wodurch Lizenzgebühren eingespart werden können.


\subsection{Entwicklungsprozess}
\label{sec:Entwicklungsprozess}
Die Entwicklung dieses Projekts folgte dem erweiterten Wasserfallmodell. 
Die Wahl dieses Modells ergab sich aus der Tatsache,
dass kein großes Entwicklerteam parallel an verschiedenen Teilaufgaben arbeitete 
und der Zeitrahmen des Projekts eine schlanke Methodologie begünstigte.
Die Entscheidung für das erweiterte Wasserfallmodell ermöglichte mir die Flexibilität, 
bei Bedarf Rücksprünge in vorherige Phasen einzuräumen und neue Erkenntnisse 
in die bereits durchlaufenen Phasen einfließen zu lassen. 
