% !TEX root = ../Projektdokumentation.tex
\section{Fazit} 
\label{sec:Fazit}

\subsection{Soll-/Ist-Vergleich}
\label{sec:SollIstVergleich}
Das Projektziel wurde nicht vollständig erreicht. Die Distribution über eine 
ausführbare Anwendung brachte einige Komplikationen mit sich. Der Datenaustausch 
zwischen Anwendung und \ac{LLaMA}-2 Model ist bei der Erstellung trotz referenzierter
Daten eines Application Bundles verloren gegangen. Weitere Recherche zu betreiben, 
wie sich dieses Problem im konkreten Anwendungsfall lösen lässt, war aufgrund der 
zeitlichen Begrenzung dieses Projekts nicht möglich. Die Anwendung konnte ebenfalls 
nicht in Produktion gehen, da die Genauigkeit des \ac{LLaMA}-2 Model nicht messbar war. 



\paragraph{Soll-/Ist-Vergleich (verkürzt)}
Die Zeitplanung konnte bis auf wenige Ausnahmen eingehalten werden.
\tabelle{Soll-/Ist-Vergleich}{tab:Vergleich}{Zeitnachher.tex}

\subsection{Lessons Learned}
\label{sec:LessonsLearned}

\paragraph{Recherche und Planung:}
Die Bedeutung einer gründlichen Recherche und Planung wurde deutlich. 
Dies gilt besonders, wenn mit neuen Technologien gearbeitet wird, 
die nicht im täglichen Einsatz des Unternehmens sind. 
Zukünftige Projekte könnten von einer umfassenderen Vorrecherche profitieren.

\paragraph{Neue Technologien:} 
Der Umgang mit neuen Technologien, wie den Generativen Transformern, 
kann Herausforderungen mit sich bringen. Es ist wichtig, 
genügend Zeit für das Erlernen und Verstehen dieser Technologien einzuplanen.

\paragraph{Verwendete Programmiersprache:} 
Der Einsatz von Python, auch wenn nicht aktiv von 
unseren Unternehmen verwendet, zeigt, dass die Verwendung einer breit akzeptierten
und unterstützten Programmiersprache Vorteile bringt. Zukünftige Projekte 
könnten davon profitieren, bereits im Unternehmen etablierte Sparchen zu bevorzugen. 

\paragraph{Frühzeitige Einbindung von Informationen:} 
Informationen über neue Technologien sollten frühzeitig in die Planung einbezogen werden. 
Dies ermöglicht es, besser auf mögliche Hindernisse vorbereitet zu sein.

\subsection{Ausblick}
\label{sec:Ausblick}
Der Ausblick auf die Weiterentwicklung der Anwendung zeigt vielversprechende Potenziale. 
Mit einer Erweiterung der Anwendungsfälle und einer verbesserten 
Präzision bei der Generierung von Antworten könnte das Projekt einen erheblichen 
Mehrwert für das Unternehmen bieten und somit in den operativen Betrieb übergehen.

Ein vielversprechender Anwendungsfall könnte darin bestehen, Echtzeitdaten in die Anwendung zu 
integrieren, insbesondere in Bezug auf die Verwaltungsoberfläche des Unternehmens. 
In der Veranstaltungsbranche, die oft von schnellen Veränderungen und Unvorhersehbarkeiten geprägt ist, 
kann die Fähigkeit, auf Echtzeitereignisse zu reagieren, entscheidend für den Erfolg sein. 
Sollte die Anwendung in der Lage ist, diese Echtzeitereignisse präzise zu klassifizieren, 
eröffnet sich ein breites Spektrum an lösbaren Support-Fällen.

Die kontinuierliche Verbesserung der Anwendung im Hinblick auf neue Anforderungen und Erkenntnisse
könnte dazu beitragen, dass die Lösung nicht nur den aktuellen Bedürfnissen gerecht wird, 
sondern auch zukünftige Herausforderungen erfolgreich bewältigen kann.

\clearpage
