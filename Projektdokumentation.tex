%-----------------------------------------------------------------------------------------
% Autor dieser Vorlage:
% Stefan Macke (http://fachinformatiker-anwendungsentwicklung.net)
% Permalink zur Vorlage: http://fiae.link/LaTeXVorlageFIAE
%
% Sämtliche verwendeten Abbildungen, Tabellen und Listings stammen von Dirk Grashorn.
%
% Lizenz: Creative Commons 4.0 Namensnennung - Weitergabe unter gleichen Bedingungen
% -----------------------------------------------------------------------------------------

\documentclass[
	ngerman,
	toc=listof, % Abbildungsverzeichnis sowie Tabellenverzeichnis in das Inhaltsverzeichnis aufnehmen
	footnotes=multiple, % Trennen von direkt aufeinander folgenden Fußnoten
	parskip=half, % vertikalen Abstand zwischen Absätzen verwenden anstatt horizontale Einrückung von Folgeabsätzen
	numbers=noendperiod % Den letzten Punkt nach einer Nummerierung entfernen (nach DIN 5008)
]{scrartcl}
\pdfminorversion=5 % erlaubt das Einfügen von pdf-Dateien bis Version 1.7, ohne eine Fehlermeldung zu werfen (keine Garantie für fehlerfreies Einbetten!)
\usepackage[utf8]{inputenc} % muss als erstes eingebunden werden, da Meta/Packages ggfs. Sonderzeichen enthalten
\usepackag{glossaries}
% !TEX root = Projektdokumentation.tex

% Hinweis: der Titel muss zum Inhalt des Projekts passen und den zentralen Inhalt des Projekts deutlich herausstellen
\newcommand{\titel}{Entwicklung eines autonomen Support-Bots}
\newcommand{\untertitel}{Anwendung zur Teilautomatisierung des Kunden-Supports}
\newcommand{\kompletterTitel}{\titel{} -- \untertitel}

\newcommand{\autorName}{Michele Ayadi}
\newcommand{\autorAnschrift}{Steprathstraße 14}
\newcommand{\autorOrt}{51103 Köln}

\newcommand{\betriebLogo}{LogoBetrieb.pdf}
\newcommand{\betriebName}{ticket i/O}
\newcommand{\betriebAnschrift}{Im Zollhafen 2-4}
\newcommand{\betriebOrt}{ 50678 Köln}

\newcommand{\ausbildungsberuf}{Fachinformatiker für Anwendungsentwicklung}
\newcommand{\betreff}{Dokumentation zur betrieblichen Projektarbeit}
\newcommand{\pruefungstermin}{November 2023}
\newcommand{\abgabeOrt}{Köln}
\newcommand{\abgabeTermin}{16.11.2023}
 % Metadaten zu diesem Dokument (Autor usw.)
\input{Allgemein/Packages} % verwendete Packages
% !TEX root = ../Projektdokumentation.tex

% Seitenränder -----------------------------------------------------------------
\setlength{\topskip}{\ht\strutbox} % behebt Warnung von geometry
\geometry{a4paper,left=25mm,right=15mm,top=25mm,bottom=35mm}

\usepackage[
	automark, % Kapitelangaben in Kopfzeile automatisch erstellen
	headsepline, % Trennlinie unter Kopfzeile
	ilines % Trennlinie linksbündig ausrichten
]{scrlayer-scrpage}

% Kopf- und Fußzeilen ----------------------------------------------------------
\pagestyle{scrheadings}
% chapterpagestyle gibt es nicht in scrartcl
%\renewcommand{\chapterpagestyle}{scrheadings}
\clearscrheadfoot

% Kopfzeile
\renewcommand{\headfont}{\normalfont} % Schriftform der Kopfzeile
\ihead{\large{\textsc{\titel}}\\ \small{\untertitel} \\[2ex] \textit{\headmark}}
\chead{}
\ohead{\includegraphics[scale=0.5]{\betriebLogo}}
\setlength{\headheight}{15mm} % Höhe der Kopfzeile
%\setheadwidth[0pt]{textwithmarginpar} % Kopfzeile über den Text hinaus verbreitern (falls Logo den Text überdeckt)

% Fußzeile
\ifoot{\autorName}
\cfoot{}
\ofoot{\pagemark}

% Überschriften nach DIN 5008 in einer Fluchtlinie
% ------------------------------------------------------------------------------

% Abstand zwischen Nummerierung und Überschrift definieren
% > Schön wäre hier die dynamische Berechnung des Abstandes in Abhängigkeit
% > der Verschachtelungstiefe des Inhaltsverzeichnisses
\newcommand{\headingSpace}{1.5cm}

% Abschnittsüberschriften im selben Stil wie beim Inhaltsverzeichnis einrücken
\renewcommand*{\othersectionlevelsformat}[3]{
  \makebox[\headingSpace][l]{#3\autodot}
}

% Für die Einrückung wird das Paket tocloft benötigt
%\cftsetindents{chapter}{0.0cm}{\headingSpace}
\cftsetindents{section}{0.0cm}{\headingSpace}
\cftsetindents{subsection}{0.0cm}{\headingSpace}
\cftsetindents{subsubsection}{0.0cm}{\headingSpace}
\cftsetindents{figure}{0.0cm}{\headingSpace}
\cftsetindents{table}{0.0cm}{\headingSpace}


% Allgemeines
% ------------------------------------------------------------------------------

\onehalfspacing % Zeilenabstand 1,5 Zeilen
\frenchspacing % erzeugt ein wenig mehr Platz hinter einem Punkt

% Schusterjungen und Hurenkinder vermeiden
\clubpenalty = 10000
\widowpenalty = 10000
\displaywidowpenalty = 10000

% Quellcode-Ausgabe formatieren
\lstset{numbers=left, numberstyle=\tiny, numbersep=5pt, breaklines=true}
\lstset{emph={square}, emphstyle=\color{red}, emph={[2]root,base}, emphstyle={[2]\color{blue}}}

\counterwithout{footnote}{section} % Fußnoten fortlaufend durchnummerieren
\setcounter{tocdepth}{3} % im Inhaltsverzeichnis werden die Kapitel bis zum Level der subsubsection übernommen
\setcounter{secnumdepth}{3} % Kapitel bis zum Level der subsubsection werden nummeriert

% Aufzählungen anpassen
\renewcommand{\labelenumi}{\arabic{enumi}.}
\renewcommand{\labelenumii}{\arabic{enumi}.\arabic{enumii}.}
\renewcommand{\labelenumiii}{\arabic{enumi}.\arabic{enumii}.\arabic{enumiii}}

% Tabellenfärbung:
\definecolor{heading}{rgb}{0.42, 0.35, 0.8}
\definecolor{odd}{rgb}{0.9,0.9,0.9}
 % Definitionen zum Aussehen der Seiten
% Abkürzungen
x % eigene allgemeine Befehle, die z.B. die Arbeit mit LaTeX erleichtern
% Abkürzungen
x % eigene projektspezifische Befehle, z.B. Abkürzungen usw.

\begin{document}

% ---------------------------------------------------------------------------

\phantomsection
\thispagestyle{empty}
\pdfbookmark[1]{Eidesstattliche Erklärung}{ihkdeckblatt}
\includegraphicsKeepAspectRatio{DeckblattIHK}{1}
\cleardoublepage


\phantomsection
\thispagestyle{plain}
\pdfbookmark[1]{Deckblatt}{deckblatt}
% !TEX root = Projektdokumentation.tex
\begin{titlepage}

\begin{center}
\includegraphics[scale=0.25]{LogoIHK.pdf}\\[1ex]
\Large{Abschlussprüfung \pruefungstermin}\\[3ex]

\Large{\ausbildungsberuf}\\
\LARGE{\betreff}\\[4ex]

\huge{\textbf{\titel}}\\[1.5ex]
\Large{\textbf{\untertitel}}\\[4ex]

\normalsize
Abgabetermin: \abgabeOrt, den \abgabeTermin\\[3em]
\textbf{Prüfungsbewerber:}\\
\autorName\\
\autorAnschrift\\
\autorOrt\\[5ex]


\textbf{Ausbildungsbetrieb:}\\
\betriebName\\
\betriebAnschrift\\
\betriebOrt\\[5em]
\includegraphics[scale=0.6]{\betriebLogo}\\[2ex]
\end{center}
\end{titlepage}
\cleardoublepage

% Preface --------------------------------------------------------------------
\phantomsection
\pagenumbering{arabic}
\pdfbookmark[1]{Inhaltsverzeichnis}{inhalt}
\tableofcontents

\cleardoublepage

\phantomsection
\listoffigures
\cleardoublepage

\phantomsection
\listoftables
\cleardoublepage

\phantomsection
\lstlistoflistings
\cleardoublepage

\newcommand{\abkvz}{Abkürzungsverzeichnis}
\renewcommand{\nomname}{\abkvz}
\section*{\abkvz}
\markboth{\abkvz}{\abkvz}
\addcontentsline{toc}{section}{\abkvz}
% !TEX root = Projektdokumentation.tex

% Es werden nur die Abkürzungen aufgelistet, die mit \ac definiert und auch benutzt wurden. 
%
% \acro{VERSIS}{Versicherungsinformationssystem\acroextra{ (Bestandsführungssystem)}}
% Ergibt in der Liste: VERSIS Versicherungsinformationssystem (Bestandsführungssystem)
% Im Text aber: \ac{VERSIS} -> Versicherungsinformationssystem (VERSIS)

% Hinweis: allgemein bekannte Abkürzungen wie z.B. bzw. u.a. müssen nicht ins Abkürzungsverzeichnis aufgenommen werden
% Hinweis: allgemein bekannte IT-Begriffe wie Datenbank oder Programmiersprache müssen nicht erläutert werden,
%          aber ggfs. Fachbegriffe aus der Domäne des Prüflings (z.B. Versicherung)

% Die Option (in den eckigen Klammern) enthält das längste Label oder
% einen Platzhalter der die Breite der linken Spalte bestimmt.
\begin{acronym}[WWWWW]
	\acro{API}{Application Programming Interface}
	\acro{GPT}{Generative Pre-trained Transformers}
	\acro{LLM}{Large Langue Model}
	\acro{GGUF}{GPT-Generated Unified Format}
	\acro{GB}{Gigabyte}
	\acro{RAM}{Random Access Memory}
	\acro{CPU}{Central Processing Unit}
	\acro{LLaMA}{Large Language Model Meta AI}
\end{acronym}

\clearpage

\newcommand{\gl}{Glossar}
\renewcommand{\nomname}{\gl}
\section*{\gl}
\markboth{\gl}{\gl}
\addcontentsline{toc}{section}{\gl}
% !TEX root = Projektdokumentation.tex
\makeglossaries
\newglossaryentry{LLaMA}
{
    Name=LLaMA,
    Beschreibung={Ein von Meta entwickeltes generatives Sprachmodel, das auf Basis von künstlichen neuronalen Netzwerken natürliche Sprache generiert. }
}

\newglossaryentry{Large Language Model}
{
    Name=Large Language Model,
    Beschreibung={Ein generatives Sprachmodel}
}

\newglossaryentry{LangChain}
{
    Name=LangChain,
    Beschreibung={Eine Bibliothek um die Kommunikation mit generatives Sprachmodellen aufzureichern }
}

\newglossaryentry{LLaMA.cpp}
{
    Name=LLaMA.cpp,
    Beschreibung={Eine Bibliothek um das LLaMA-Model mit einer 4-Bit Tokenlänge auf einem MacBook auszuführen.}
}

\newglossaryentry{Generative pre-trained transformer}
{
    Name=Generative pre-trained transformer,
    Beschreibung={Eine Art von generativen Sprachmodellen, welche vor ihrer Nutzung mit einem riesigen Datensatz trainiert wurden.}
}

\newglossaryentry{Halluzinationen}
{
    Name=Halluzinationen,
    Beschreibung={Dazu Fehlinterpretieren oder Dazuerfinden von Text eines generatives Sprachmodelles}
}

\newglossaryentry{GPT-Generated Unified Format}
{
    Name=GPT-Generated Unified Format,
    Beschreibung={Vereinheitlichtes Format um mit generatives Sprachmodellen zu inferieren}
}

\newglossaryentry{Token}
{
    Name=Token,
    Beschreibung={Im Rahmen von generativen Sprachmodellen sind es Zeichenketten welchen von diesen Modellen generiert oder entgegengenommen werden.}
}

\newglossaryentry{Zendesk}
{
    Name=Zendesk,
    Beschreibung={Kundendienst Software, welche es ermöglicht Ticket basiert Kundenanliegen zu lösen}
}

\newglossaryentry{Makro}
{
    Name=Makro,
    Beschreibung={Eine E-Mail-Vorlage welche in Zendesk erstellt werden kann, um schneller auf Kundenanliegen zu reagieren.}
}

\newglossaryentry{POST-Request}
{
    Name=POST-Request,
    Beschreibung={Eine auf dem HyperText Transfer Protocol basierende Anfrage, welche Daten zu einem Server sendet. }
}

\newglossaryentry{API-Token}
{
    Name=API-Token,
    Beschreibung={Ein alphanumerische Zeichen welches genutzt wird um sich bei Anwendungsschnittstelle zu authentifizieren.}
}

\glsaddall
\glsaddallunused
\printglossary


\clearpage


% Inhalt ---------------------------------------------------------------------
\pagenumbering{arabic}
\input{Inhalt.tex}


% Anhang ---------------------------------------------------------------------
\appendix
\pagenumbering{arabic}
% !TEX root = Projektdokumentation.tex
\section{Anhang}
\subsection{Detaillierte Zeitplanung}
\label{app:Zeitplanung}
\vspace{20mm}
\begin{*figure}
\centering
\renewcommand{\floatpagefraction}{0.8}
\includegraphicsKeepAspectRatio{Ganttdiagramm.png}{0.8}
\end{figure*}
\clearpage

\subsection{Lastenheft (Auszug)}
\label{app:Lastenheft}
Es folgt ein Auszug aus dem Lastenheft mit Fokus auf die Anforderungen:

Die Anwendung muss folgende Anforderungen erfüllen: 
\begin{enumerate}[itemsep=0em,partopsep=0em,parsep=0em,topsep=0em]
\item Ticket-Einstufung und Kompetenzbewertung:
	\begin{enumerate}
	\item Die Anwendung muss Support-Tickets eigenständig analysieren undkorrekt in Kategorien oder Schwierigkeitsgrade einstufen können.
	\item Sie sollte die eigene Kompetenz bei der Lösung von Support-Ticketsbewerten und sicherstellen, dass sie nur Tickets bearbeitet, für die sie ausreichend qualifiziert ist.
	\end{enumerate}
\item Zugriff auf E-Mail-Vorlagen:
	\begin{enumerate}
	\item Die Anwendung sollte in der Lage sein, auf eine Vielzahl von vordefinierten E-Mail-Vorlagen für den Kunden-Support zuzugreifen und diese bei Bedarf in die Kommunikation einzubinden.
	\end{enumerate}
\item Verhaltenstransparenz und Export:
	\begin{enumerate}
	\item Die Anwendung sollte in der Lage sein, ihre Aktionen und Entscheidungen in einem exportierbaren Format bereitzustellen, um die Nachvollziehbarkeit und Analyse des Bot-Verhaltens zu ermöglichen.
	\end{enumerate}
\item Bedienbarkeit
	\begin{enumerate}
	\item Die Anwendung sollte als ausführbares Programm bereitgestellt werden.
	\end{enumerate}
\end{enumerate}


\clearpage

\subsection{Anwendungsfalldiagramm}
\label{app:UseCase}
\vspace{20mm}
\begin{figure}[htb]
\centering
\includegraphicsKeepAspectRatio{Anwendungsfalldiagramm.pdf}{1}
\caption{Anwendungsfalldiagramm}
\end{figure}
\clearpage


\subsection{Klassendiagramm}
\label{app:Klassendiagramm}
\vspace{20mm}
\begin{figure}[htb]
\centering
\includegraphicsKeepAspectRatio{Klassendiagramm.pdf}{1}
\caption{Klassendiagramm}
\end{figure}
\clearpage


\subsection{Pflichtenheft (Auszug)}
\label{app:Pflichtenheft}

\subsubsection*{Zielbestimmung}
Das vorliegende Pflichtenheft stellt die spezifischen Anforderungen an die Anwendung ``SupportBot'' dar. Diese Anforderungen dienen als Grundlage für die Implementierung und Entwicklung der Anwendung.
		

\begin{enumerate}[itemsep=0em,partopsep=0em,parsep=0em,topsep=0em]
\item Musskriterien
	\begin{enumerate}
		\item Ticket-Einstufung und Kompetenzbewertung
		\begin{itemize}
			\item Die Anwendung muss in der Lage sein, Support-Tickets eigenständig zu analysieren und korrekt in bestimmte Kategorien oder Schwierigkeitsgrade einzustufen. Dies dient dazu, die Priorität und Bearbeitungshierarchie festzulegen.
			\item Die Anwendung sollte die eigene Kompetenz bei der Lösung von Support-Tickets bewerten und sicherstellen, dass sie nur Tickets bearbeitet, für die sie ausreichend qualifiziert ist. Dies gewährleistet eine effiziente Ticketzuweisung.
		\end{itemize}
		\item Zugriff auf E-Mail-Vorlagen
			\begin{itemize}
			\item Die Anwendung sollte in der Lage sein, auf eine Vielzahl von vordefinierten E-Mail-Vorlagen für den Kunden-Support zuzugreifen. Bei Bedarf kann sie diese Vorlagen in die Kommunikation mit den Kunden einbinden. Dies ermöglicht eine konsistente und effektive Kommunikation.
			\end{itemize}
		\item Verhaltenstransparenz und Export
			\begin{itemize}
			\item Die Anwendung sollte die Fähigkeit besitzen, ihre Aktionen und Entscheidungen in einem exportierbaren Format bereitzustellen. Dies dient der Nachvollziehbarkeit und Analyse des Bot-Verhaltens. Das exportierbare Format sollte für Analysezwecke geeignet sein.
			\end{itemize}
		\item Bedienbarkeit
			\begin{itemize}
			\item Die Anwendung wird als ausführbares Programm bereitgestellt. Dies ermöglicht die einfache Installation und Nutzung auf den Rechnern der Kundensupport-Mitarbeiter.
			\end{itemize}
	\end{enumerate}
\end{enumerate}



\clearpage
\subsection{Prozesskostenrechnung}
\label{app:Prozesskosten}
\begin{figure}[htb]
\centering
\includegraphicsKeepAspectRatio{prozesskosten.png}{0.9}
\caption{Detaillierte Prozesskostenrechnung}
\end{figure}


\clearpage
\subsection{Amortisationsrechnung}
\label{app:Amortisationsrechnung}
\vspace{20mm}
\begin{figure}[htb]
\centering
\includegraphicsKeepAspectRatio{amortisationsrechnung.png}{1}
\caption{Detaillierte Amortisationsrechnung}
\end{figure}


\clearpage

\subsection{Screenshots des Quellcodes}
\label{Screenshots}
\vspace{20mm}
\begin{figure}[htb]
\centering
\includegraphicsKeepAspectRatio{zendesk.png}{1}
\caption{Auschnitt ZendeskService-Klasse}
\end{figure}
\clearpage
\vspace{20mm}
\begin{figure}[htb]
\centering
\includegraphicsKeepAspectRatio{supportagent.png}{1}
\caption{Auschnitt SupportAgent-Klasse}
\end{figure}
\clearpage

\clearpage
\subsection{Testfall SupportAgent Klasse}
\label{app:Test}
\vspace{20mm}
\newcommand{\listingsttfamily}{\fontfamily{NotoSansMono-TLF}\small}
\lstinputlisting[language=python, caption={Testfall in Python}]{Listings/test_support_agent.py}
\clearpage
\begin{figure}[htb]
\centering
\includegraphicsKeepAspectRatio{gelöstes_ticket.png}{1}
\caption{Ein Screenshot eines gelösten Support-Falles}
\end{figure}








\end{document}
